\documentclass[12pt]{article}
\usepackage{stmaryrd}
\usepackage{graphicx}
\usepackage[utf8]{inputenc}

\usepackage[french]{babel}
\usepackage[T1]{fontenc}
\usepackage{hyperref}
\usepackage{verbatim}

\usepackage{color, soul}

\usepackage{pgfplots}
\pgfplotsset{compat=1.15}
\usepackage{mathrsfs}

\usepackage{amsmath}
\usepackage{amsfonts}
\usepackage{amssymb}
\usepackage{tkz-tab}

\usepackage{tikz}
\usetikzlibrary{arrows, shapes.geometric, fit}


\usepackage[margin=2cm]{geometry}

\usepackage{makecell}  % Ajoutez ceci dans le préambule
\begin{document}

\begin{minipage}{0.8\textwidth}
	Pathé BA                          
\end{minipage}
\begin{minipage}{0.8\textwidth}
	BAC-L 2024
\end{minipage}

\begin{center}
\textbf{{\underline{\textcolor{red}{Premier Groupe}}}}
\end{center}

\section*{\textcolor{red}{\underline{Exercice 1} (6 points) :}}
Pour chaque item choisir la bonne réponse dans la colonne de droite, sachant qu'une seule réponse est correcte.

Chaque bonne reponse rapporte \textbf{ 01 point}.

\begin{tabular}{|c|c|}
\hline
\textbf{ITEMS}  & \textbf{REPONSES}\\
\hline
1. Si les fonctions $f$ et $g$ sont dfinies par $f(x)=\sqrt{x^{2}-9}$ et $g(x)=\frac{x+3}{x-1}$, alors: &
\makecell{a.$(g\circ f)(2)=4$\\ b.$(f\circ g)(3)=-3$\\c.$(g\circ f)(3)=-3$}\\
\hline
\end{tabular}

\end{document}