\documentclass[12pt]{article}
\usepackage{stmaryrd}
\usepackage{graphicx}
\usepackage[utf8]{inputenc}

\usepackage[french]{babel}
\usepackage[T1]{fontenc}
\usepackage{hyperref}
\usepackage{verbatim}

\usepackage{color, soul}

\usepackage{pgfplots}
\pgfplotsset{compat=1.15}
\usepackage{mathrsfs}

\usepackage{amsmath}
\usepackage{amsfonts}
\usepackage{amssymb}
\usepackage{tkz-tab}

\usepackage{tikz}
\usetikzlibrary{arrows, shapes.geometric, fit}


\usepackage[margin=2cm]{geometry}

\usepackage{makecell}  % Ajoutez ceci dans le préambule
\begin{document}

\begin{minipage}{0.8\textwidth}
	Pathé BA                          
\end{minipage}
\begin{minipage}{0.8\textwidth}
	BAC-L 2024
\end{minipage}

\begin{center}
\textbf{{\underline{\textcolor{red}{Premier Groupe}}}}
\end{center}

\section*{\textcolor{red}{\underline{Exercice 1} (4 points) :}}
Pour chaque item choisir la bonne réponse dans la colonne de droite, sachant qu'une seule réponse est correcte.

Chaque bonne reponse rapporte \textbf{( 01 point)}.

\begin{tabular}{|c|c|}
\hline
\textbf{ITEMS} & \textbf{REPONSES}\\
\hline
\makecell[l]{\textbf{1.} Si les fonctions $f$ et $g$ sont définies par \\ $f(x)=\sqrt{x^{2}-9}$ et $g(x)=\frac{x+3}{x-1}$, alors:} &
\makecell[l]{\textbf{a.} $(g\circ f)(2)=4$ \\ \textbf{b.} $(f\circ g)(3)=-3$ \\ \textbf{c.} $(g\circ f)(3)=-3$}\\
\hline
\makecell[l]{\textbf{2.} Si la fonction $f$ est définie par \\ $f(x)=\frac{e^{x}}{1+e^{x}}$, alors:} &
\makecell[l]{\textbf{a.} La dérivée de $f$ est $f'(x)=\frac{1}{\left(1+e^{x}\right)^{2}}$ \\ \textbf{b.} Une primitive de $f$ sur $\mathbb{R}$ est\\ $F(x)=\ln(1+e^{x})$ \\ \textbf{c.} $f$ est définie sur $\mathbb{R}\setminus\left\lbrace -1\right\rbrace $}\\ 
\hline
\makecell[l]{\textbf{3.} Une radio a commencé à émettre en l'an $2000$\\ avec $5000$ auditeurs. Chaque année elle perd \\ $20\%$ de ses auditeurs, mais elle en accueille $4000$\\ nouveaux. Soit $U_{n}$ le nombre d'auditeurs de la\\ radio en  l'an $2000+n$: } & \makecell[l]{\textbf{a.} $U_{n+1}=0,8U_{n}+4000$. \\ 
\textbf{b.} $U_{n+1}=0,2U_{n}+4000$. \\ \textbf{c.} $U_{n+1}=0,8U_{n}+5000$. }\\
\hline
\makecell[l]{\textbf{3.} On choisit au hasard, successivement et sans\\ remises $3$ jetons d'une caisse qui contient 1 jeton\\ vert, 2 jetons jaunes et 3 jetons rouges.\\ La proabilité de tirer 3 jetons de couleurs différentes est:} & 
\makecell[l]{\textbf{a.} $\frac{1}{20}$. \\ 
\textbf{b.} $\frac{3}{10}$. \\ \textbf{c.} $\frac{ A_{1}^{1}\times A_{2}^{1}\times A_{3}^{1}}{ A_{6}^{3} }$. }\\
\hline
\end{tabular}
\section*{\textcolor{red}{\underline{Exercice 1} (6 points) :}}
\begin{enumerate}
\item Résoudre dans $\mathbb{R}$:
	\begin{enumerate}
	\item L'équation $\ln(2x+1)+\ln(x-1)=\ln2$. \textbf{(01,5 point)}
	\item L'inéquation $e^{2x}-3e^{x}-4 \leq 0$. \textbf{(01 point)}
	\end{enumerate}
\item \[ \text{Résoudre dans } \mathbb{R}^{2} \text{ le système }
\begin{cases}
\ln x + \ln y = \ln 2\\
e^{x}\times e^{y}=e^{3}
\end{cases}  \textbf{(01,5 point)}\]
\item Résoudre dans $\mathbb{R}^{3}$ :
\begin{enumerate}
\item \[ \text{Le système }
\begin{cases}
x-y+z = -2\\
2x+y-2z = 6\\
x-3y-z=-4
\end{cases}  \textbf{(01 point)}\]
\item \[ \text{En déduire la résolution du système }
\begin{cases}
\ln x-\ln y+\ln z = -2\\
\ln x^{2}+\ln y-\ln z^{2} = 6\\
\ln x-\ln y^{3}-\ln z=-4
\end{cases}  \textbf{(01 point)}\]
\end{enumerate}
\end{enumerate}
\section*{\textcolor{red}{\underline{Exercice 2} (10 points) :}}
\subsubsection*{\textcolor{red}{\underline{PROBLÈME}:}}
Soit la fonction numérique $f$ de la variable réelle $x$, définie par $f(x)=\frac{1+\ln x}{x}$ et $(\mathscr{Cf})$ et sa courbe représentative dans un repère orthonormé $(O;\vec{i};\vec{j})$ d'unité graphique 1 cm.
\begin{enumerate}
\item 
\begin{itemize}
\item[a.] Déterminer l'ensemble  de définition $Df$ de f,puis calculer les limites aux bornes  de $Df$.
\item[b.] Iterpréter graphiquement les résultats obtenus. \textbf{(01 point)}
\end{itemize}
\item 
\begin{itemize}
\item[a.] Résoudre dans $\mathbb{R}$ l'inéquation $\ln x \leq 0$. \textbf{(0,75 point)}
\item[b.] Montrer que la dérivée de $f$ est définie pour tout $x>0$, par $f'(x)=\frac{-\ln x}{x}$, puis étudier son signe.\textbf{(01 point)}
\item[c.] Dresser le tableau de variation de $f$ \textbf{(01 point)}
\end{itemize}
\item Etudier $(\mathscr{Cf})$ l'intersection de avec l'axe des abscisses. \textbf{(01 point)}
\item Tracer la courbe de $(\mathscr{Cf})$.  \textbf{(01,75 point)}
\item Soit  la fonction F définie par $F(x)=\frac{1}{2}(\ln x)^{2}+\ln x.$
\begin{itemize}
\item[a.] Montrer que $F$ est une primitive de $f$ dans $]0; +\infty[$ \textbf{(01 point)}
\item[b.] Calculer en $cm^{2}$ l'aire $\mathscr{A}$ du domaine plan délimité par la courbe $(\mathscr{Cf})$, l'axe des abscisses, ainsi que les droites $(D_{1}): x=1$ et $(D_{2}): x=e$. \textbf{(01 point)}
\end{itemize}
\end{enumerate}
\end{document}