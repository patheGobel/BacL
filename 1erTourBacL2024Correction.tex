\documentclass[12pt]{article}
\usepackage{stmaryrd}
\usepackage{graphicx}
\usepackage[utf8]{inputenc}

\usepackage[french]{babel}
\usepackage[T1]{fontenc}
\usepackage{hyperref}
\usepackage{verbatim}

\usepackage{color, soul}

\usepackage{pgfplots}
\pgfplotsset{compat=1.15}
\usepackage{mathrsfs}

\usepackage{amsmath}
\usepackage{amsfonts}
\usepackage{amssymb}
\usepackage{tkz-tab}

\usepackage{tikz}
\usetikzlibrary{arrows, shapes.geometric, fit}


\usepackage[margin=2cm]{geometry}
\usepackage{makecell}  % Ajoutez ceci dans le préambule
\begin{document}

\begin{minipage}{0.8\textwidth}
	Pathé BA                          
\end{minipage}
\begin{minipage}{0.8\textwidth}
	BAC 2024
\end{minipage}

\begin{center}
\textbf{{\underline{\textcolor{green}{Premier Groupe Correction}}}}
\end{center}
\section*{\textcolor{green}{\underline{Correction Exercice 1} (4 points) :}}
Pour chaque item choisir la bonne réponse dans la colonne de droite, sachant qu'une seule réponse est correcte.

Chaque bonne réponse rapporte \textbf{( 01 point)}.

\begin{tabular}{|c|c|}
\hline
\textbf{ITEMS} & \textbf{REPONSES}\\
\hline
\makecell[l]{\textbf{1.} Si les fonctions $f$ et $g$ sont définies par \\ $f(x)=\sqrt{x^{2}-9}$ et $g(x)=\frac{x+3}{x-1}$, alors:} &
\makecell[l]{\textbf{a.} $(g\circ f)(2)=4$ \\ \textbf{b.} $(f\circ g)(3)=-3$ \\ \textbf{c.} $(g\circ f)(3)=-3$}\\
\hline
\makecell[l]{\textbf{2.} Si la fonction $f$ est définie par \\ $f(x)=\frac{e^{x}}{1+e^{x}}$, alors:} &
\makecell[l]{\textbf{a.} La dérivée de $f$ est $f'(x)=\frac{1}{\left(1+e^{x}\right)^{2}}$ \\ \textbf{b.} Une primitive de $f$ sur $\mathbb{R}$ est\\ $F(x)=\ln(1+e^{x})$ \\ \textbf{c.} $f$ est définie sur $\mathbb{R}\setminus\left\lbrace -1\right\rbrace $}\\ 
\hline
\makecell[l]{\textbf{3.} Une radio a commencé à émettre en l'an $2000$\\ avec $5000$ auditeurs. Chaque année elle perd \\ $20\%$ de ses auditeurs, mais elle en accueille $4000$\\ nouveaux. Soit $U_{n}$ le nombre d'auditeurs de la\\ radio en  l'an $2000+n$: } & \makecell[l]{\textbf{a.} $U_{n+1}=0,8U_{n}+4000$. \\ 
\textbf{b.} $U_{n+1}=0,2U_{n}+4000$. \\ \textbf{c.} $U_{n+1}=0,8U_{n}+5000$. }\\
\hline
\makecell[l]{\textbf{4.} On choisit au hasard, successivement et sans\\ remises $3$ jetons d'une caisse qui contient 1 jeton\\ vert, 2 jetons jaunes et 3 jetons rouges.\\ La proabilité de tirer 3 jetons de couleurs différentes est:} & 
\makecell[l]{\textbf{a.} $\frac{1}{20}$. \\ 
\textbf{b.} $\frac{3}{10}$. \\ \textbf{c.} $\frac{ A_{1}^{1}\times A_{2}^{1}\times A_{3}^{1}}{ A_{6}^{3} }$. }\\
\hline
\end{tabular}\\

\textcolor{green}{\textbf{1.} Si les fonctions $f$ et $g$ sont définies par  $f(x)=\sqrt{x^{2}-9}$ et $g(x)=\frac{x+3}{x-1}$, alors:}

\textbf{a.} $(g\circ f)(2)=4$

\[
(g\circ f)(2)=g(f(2)):
\begin{cases}
f(2)=\sqrt{2^{2}-9} \text{ impossible }\\
\end{cases}
\]

\textbf{b.} $(f\circ g)(3)=-3$
 \[
(f\circ g)(3)=f(g(3)):
\begin{cases}
g(3)=\frac{6}{2}=3\\
f(g(3))=f(3)=\sqrt{3^{2}-9}=0 \text{ Pas une bonne réponse }
\end{cases}
\]
\textbf{c.} $(g\circ f)(3)=-3$
\[
(g\circ f)(3)=g(f(3)):
\begin{cases}
f(3)=\sqrt{3^{2}-9}=0\\
g(f(3))=g(0)=-3 \text{ C'est une bonne réponse }
\end{cases}
\]

Donc la bonne réponse est \textcolor{green}{\textbf{c.} $(g\circ f)(3)=-3$}\\

\textcolor{green}{ \textbf{2.} Si la fonction $f$ est définie par $f(x)=\frac{e^{x}}{1+e^{x}}$, alors: }
\newpage
\textbf{a.} La dérivée de $f$ est $f'(x)=\frac{1}{\left(1+e^{x}\right)^{2}}$

\[
f'(x)=\frac{e^{x}}{1+e^{x}}=\frac{(e^{x})'(1+e^{x})-(1+e^{x})'(e^{x})}{\left( 1+e^{x} \right)^{2}  }=\frac{(e^{x}+e^{2x})-e^{2x}}{\left( 1+e^{x} \right)^{2}}=\frac{e^{x}}{\left( 1+e^{x} \right)^{2}}
\]
\[f'(x)=\frac{e^{x}}{\left( 1+e^{x} \right)^{2}} \text{\textcolor{green} {C'est une bonne réponse }}\]
\textbf{b.} Une primitive de $f$ sur $\mathbb{R}$ est

$F(x)=\ln(1+e^{x})$
\[\text{ Vérifions si } F'(x)=f(x) \text{ en effet } F'(x)=\frac{e^{x}}{1+e^{x}} \text{ donc } F(x) \text{ est bien la primitive de } f. \]
\[\text{\textcolor{green} {C'est une bonne réponse }}\]
\textbf{c.} $f$ est définie sur $\mathbb{R}\setminus\left\lbrace -1\right\rbrace $
\[\text{\textcolor{green} {Trivial: C'est pas une bonne réponse }}\]

Donc les bonnes réponses sont \textcolor{green}{\textbf{a}, \textbf{b}}\\

\textcolor{green}{\textbf{3.} Une radio a commencé à émettre en l'an $2000$ avec $5000$ auditeurs. Chaque année elle perd  $20\%$ de ses auditeurs, mais elle en accueille $4000$ nouveaux. Soit $U_{n}$ le nombre d'auditeurs de la radio en  l'an $2000+n$:} 

\textbf{Formulation du problème :}

\begin{itemize}
\item[•] Initialement (en 2000) : $U_{0}=5000$ auditeurs.
\item[•] Évolution annuelle :
\begin{itemize}
\item[•] La radio perd $20\%$ de ses auditeurs existants : elle en conserve donc $80\%$ ou $0,8\times U_{n}$
\item[•] La radio gagne 4000 nouveaux auditeurs chaque année.
\end{itemize}
\end{itemize}
\textbf{Formule de récurrence :}

On peut exprimer le nombre d'auditeurs de la radio en l'an 2000+n (donc au bout de n années) à l'aide de la relation de récurrence suivante :$U_{n+1}=0,8\times U_{n}+4000$

où $ U_{n} $ représente le nombre d'auditeurs en l'an 2000+n

\textbf{Point de départ :} 

Pour l'année de départ, on a : $U_{0}=5000$

\textbf{Calcul des premières valeurs :}

\begin{itemize}
\item[•]Année 2001 $U_{1}=0,8\times 5000+4000=8000$
\item[•]Année 2002 $U_{1}=0,8\times 8000+4000=10400$
\end{itemize}
Ainsi de suite, on peut calculer $U_{n}$ pour chaque année.

\textbf{Étude de la suite :}

Cette suite est une suite récurrente linéaire d'ordre 1, avec une partie homogène\\
$U_{n+1}=0,8\times U_{n}+4000$ et un terme constant +4000.

\textbf{a.} $U_{n+1}=0,8U_{n}+4000$. \textbf{\textcolor{green}{C'est une bonne réponse}}
 
\textbf{b.} $U_{n+1}=0,2U_{n}+4000$. \textbf{\textcolor{green}{C'est pas une bonne réponse}}

\textbf{c.} $U_{n+1}=0,8U_{n}+5000$. \textbf{\textcolor{green}{C'est pas une bonne réponse}}
\newpage
\textcolor{green}{\textbf{4.} On choisit au hasard, successivement et sans remises $3$ jetons d'une caisse qui contient 1 jeton vert, 2 jetons jaunes et 3 jetons rouges. La proabilité de tirer 3 jetons de couleurs différentes est:}

\textbf{a.} $\frac{1}{20}$.

\[\text{\textcolor{green} {C'est une bonne réponse car une simplification de \textbf{c.} }}\]

\textbf{b.} $\frac{3}{10}$.

 \[\text{\textcolor{green} { C'est pas une bonne réponse }}\]

\textbf{c.} $\frac{ A_{1}^{1}\times A_{2}^{1}\times A_{3}^{1}}{ A_{6}^{3} }$.

\[\text{\textcolor{green} {C'est la prémière bonne réponse }}\]
 
\section*{\textcolor{green}{\underline{Correction Exercice 2} (6 points) :}}
\begin{enumerate}
\item Résolvons dans $\mathbb{R}$:
	\begin{enumerate}
	\item L'équation $\ln(2x+1)+\ln(x-1)=\ln2$. \textbf{(01,5 point)}
	
	\textbf{\underline{Domaine de validité $Dv$ :} }
	
		L'équation n'a de sens que ssi $2x+1>0$ et $x-1>0$
		
			$2x+1>0$ et $x-1>0\implies x>\frac{-1}{2}$ et $x>1\implies x\in ]1;+\infty[$

			Donc \textcolor{green} {$Dv=]1;+\infty[$}
			
	\textbf{\underline{Résolution}} 
	
\[\ln(2x+1)+\ln(x-1)=\ln2\implies \ln\left[(2x+1)(x-1)\right]=\ln2\implies \ln\left[2x^{2}-x-1\right]=\ln2\]
\[2x^{2}-x-1=2\implies 2x^{2}-x-3=0\]
\[\text{Cherchons }\Delta.\]
\[\Delta=25\]
\[x_{1}=\frac{3}{2} \text{ ; } x_{2}=-1\]
\[\text{A-t-on } x_{1} \in ]1;+\infty[ \text{ ; } x_{2} \in ]1;+\infty[ \text{ ? }\]
	
\[\text{Pour }	x_{1} \in ]1;+\infty[ \Leftrightarrow \frac{3}{2} \in ]1;+\infty[ \text{\textcolor{green} { Vrai } }\]

\[\text{Pour }	x_{2} \in ]1;+\infty[ \Leftrightarrow -1 \in ]1;+\infty[ \text{\textcolor{green} { Faux } }\]

\[\textcolor{green}{\boxed{S=\left\lbrace \frac{3}{2} \right\rbrace}}\]

\item L'inéquation $e^{2x}-3e^{x}-4 \leq 0$. \textbf{(01 point)}
\[e^{2x}-3e^{x}-4 \leq 0\]
\[\text{Posons X=}e^{x} \]
\[X^{2}-3X-4 \leq 0\]
\[\text{Cherchons }\Delta.\]
\[\Delta=25\]
\[X_{1}=4 \text{ ; } X_{2}=-1\]
\[X^{2}-3X-4 \leq 0 \implies X\in [-1;4] \implies e^{x}\in [-1;4] \implies e^{x}\in [0;4]\]
\[\text{Posons } f(x)=e^{x}.\quad f(x)\in ]0;4]\implies x\in f(]0;4])^{-1}\]
\[f(]0;4])^{-1}=]\lim_{x \to 0^{+}}\ln(x);\lim_{x \to 4}\ln(x)] \text{ car }f(x)=e^{x}\]
\[x\in ]-\infty; \ln(4)]\]
\[ \textcolor{green}{\boxed{S=\left] -\infty; \ln(4)\right] }} \]
\end{enumerate}
\item \[ \text{Résolvons dans } \mathbb{R}^{2} \text{ le système }
\begin{cases}
\ln x + \ln y = \ln 2\\
e^{x}\times e^{y}=e^{3}
\end{cases}  \textbf{(01,5 point)}\]
\[
\begin{cases}
\ln x + \ln y = \ln 2\\
e^{x}\times e^{y}=e^{3}
\end{cases}\implies
\begin{cases}
\ln x + \ln y = \ln 2\\
\ln\left( e^{x}\times e^{y}\right) =\ln\left( e^{3} \right) 
\end{cases}\implies
\begin{cases}
\ln x + \ln y = \ln 2\\
x+y = 3 
\end{cases}
\]

\[
\begin{cases}
\ln x + \ln y = \ln 2\\
x= 3-y 
\end{cases}\implies
\begin{cases}
\ln (3-y)  + \ln y = \ln 2\\
x= 3-y 
\end{cases}\implies
\begin{cases}
\ln\left[ (3-y)y\right]  = \ln 2\\
x= 3-y 
\end{cases}\implies
\]
\[
\begin{cases}
(3-y)y  = 2\\
x= 3-y 
\end{cases}\implies
\begin{cases}
y^{2}-3y-2 = 0\\
x= 3-y 
\end{cases}
\]
\[\text{En considérant } y^{2}-3y-2 = 0 \text{, Cherchons }\Delta \]
\[\Delta=17 \]
\[y=\frac{3-\sqrt{17}}{2}\quad ou \quad y=\frac{3+\sqrt{17}}{2} \]
\begin{itemize}
\item Si $y=\frac{3-\sqrt{17}}{2}$, en ramplaçant dans $x= 3-y $, ona: $x=3-\frac{3-\sqrt{17}}{2}\implies x=\frac{3+\sqrt{17}}{2}$
\item Si $y=\frac{3+\sqrt{17}}{2}$, en ramplaçant dans $x= 3-y $, ona: $x=3-\frac{3+\sqrt{17}}{2}\implies x=\frac{3-\sqrt{17}}{2}$
\end{itemize}
\[\textcolor{green}{\boxed{S=\left\lbrace \left(\frac{3+\sqrt{17}}{2}, \frac{3-\sqrt{17}}{2} \right);\left(\frac{3-\sqrt{17}}{2}, \frac{3+\sqrt{17}}{2}\right)   \right\rbrace }}\]
\item Résolvons dans $\mathbb{R}^{3}$ :
\begin{enumerate}
\item \[ \text{Le système }
\begin{cases}
x-y+z = -2\\
2x+y-2z = 6\\
x-3y-z=-4
\end{cases}  \textbf{(01 point)}
\]

\[
\begin{cases}
x-y+z = -2\quad L1\\
2x+y-2z = 6\quad L2\\
x-3y-z=-4\quad L3
\end{cases}\text{En considérant L1 comme pivot on a:}
\begin{cases}
x-y+z = -2\quad L1\\
L'2=2L1-L2\\
L'3=L1-L3
\end{cases}\text{ donc }
\]

\[
\begin{cases}
x-y+z = -2\quad L1\\
\quad -3y+4z = -10\quad L'2\\
\quad\quad 2y+2z =2\quad L'3
\end{cases} L'2 \text{ comme pivot on a:}
\begin{cases}
x-y+z = -2\quad L1\\
-3y+4z = -10\quad L'2\\
L''3=2L'2+3L'3
\end{cases}\text{donc}
\]

\[
\begin{cases}
x-y+z = -2\quad L1\\
-3y+4z = -10\quad L'2\\
\quad\quad\quad z=-1\quad L''3
\end{cases}\implies
\begin{cases}
x= 1\quad L1\\
y = 2\quad L'2\\
z=-1\quad L''3
\end{cases}
\]
\[\textcolor{green}{\boxed{S=\left\lbrace \left(1, 2, -1  \right) \right\rbrace }}\]
\item \[ \text{En déduire la résolution du système }
\begin{cases}
\ln x-\ln y+\ln z = -2\\
\ln x^{2}+\ln y-\ln z^{2} = 6\\
\ln x-\ln y^{3}-\ln z=-4
\end{cases}  \textbf{(01 point)}\]

\[
\begin{cases}
\ln x-\ln y+\ln z = -2\\
\ln x^{2}+\ln y-\ln z^{2} = 6\\
\ln x-\ln y^{3}-\ln z=-4
\end{cases} \text{ Posons }
\begin{cases}
X =\ln x\\
Y =\ln y\\
Z =\ln z
\end{cases}\implies
\begin{cases}
X-Y+Y = -2\\
2X+Z-2Z = 6\\
X-3Y-Z=-4
\end{cases}\text{ donc }
\]

\[
\begin{cases}
X= 1\\
Y= 2\\
Z=-1
\end{cases}\implies
\begin{cases}
1 =\ln x\\
2 =\ln y\\
-1 =\ln z 
\end{cases}\implies
\begin{cases}
x=e\\
y=e^{2}\\
z=e^{-1}
\end{cases}
\]
\[\textcolor{green}{\boxed{S=\left\lbrace \left(e, e^{2}, e^{-1}\right) \right\rbrace }}\]
\end{enumerate}
\end{enumerate}
\section*{\textcolor{green}{\underline{Problème (10 points):}}}
Soit la fonction numérique $f$ de la variable réelle $x$, définie par $f(x)=\frac{1+\ln x}{x}$ et $(\mathscr{Cf})$ et sa courbe représentative dans un repère orthonormé $(O;\vec{i};\vec{j})$ d'unité graphique 1 cm.
\begin{enumerate}
\item 
\begin{itemize}
\item[a.] Déterminons l'ensemble  de définition $Df$

$f$ existe ssi $x>0$

$$\textcolor{green}{\boxed{Df=\mathbb{R}^{*}_{+}}}$$

Calculons les limites aux bornes  de $Df$

Les bornes de $Df$ sont $0$ et $+\infty$
\begin{itemize}
\item En $0$
\[
\lim_{x \to 0^{+}}f(x)=\lim_{x \to 0^{+}}\frac{1+\ln x}{x}=\lim_{x \to 0^{+}}\frac{1}{x}+\frac{\ln(x)}{x}:\begin{cases}
\lim_{x \to 0^{+}}\frac{1}{x}=+\infty\\
\lim_{x \to 0^{+}}\frac{\ln(x)}{x}=0
\end{cases} \text{ Par Somme,}
\]
$$\textcolor{green}{\boxed{\lim_{x \to 0^{+}}f(x)=+\infty}}$$
%\item En $+\infty$
\end{itemize}
%\item[b.] Iterpréter graphiquement les résultats obtenus. \textbf{(01 point)}
\end{itemize}
\item 
%\begin{itemize}
%\item[a.] Résoudre dans $\mathbb{R}$ l'inéquation $\ln x \leq 0$. \textbf{(0,75 point)}
%\item[b.] Montrer que la dérivée de $f$ est définie pour tout $x>0$, par $f'(x)=\frac{-\ln x}{x}$, puis étudier son signe.\textbf{(01 point)}
%\item[c.] Dresser le tableau de variation de $f$ \textbf{(01 point)}
%\end{itemize}
%\item Etudier $(\mathscr{Cf})$ l'intersection de avec l'axe des abscisses. \textbf{(01 point)}
%\item Tracer la courbe de $(\mathscr{Cf})$.  \textbf{(01,75 point)}
%\item Soit  la fonction F définie par $F(x)=\frac{1}{2}(\ln x)^{2}+\ln x.$
%\begin{itemize}
%\item[a.] Montrer que $F$ est une primitive de $f$ dans $]0; +\infty[$ \textbf{(01 point)}
%\item[b.] Calculer en $cm^{2}$ l'aire $\mathscr{A}$ du domaine plan délimité par la courbe $(\mathscr{Cf})$, l'axe des abscisses, ainsi que les droites $(D_{1}): x=1$ et $(D_{2}): x=e$. \textbf{(01 point)}
%\end{itemize}
\end{enumerate}
\end{document}
